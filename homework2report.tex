% Created 2013-02-05 Tue 02:21
\documentclass[11pt]{article}
\usepackage[utf8]{inputenc}
\usepackage[T1]{fontenc}
\usepackage{fixltx2e}
\usepackage{graphicx}
\usepackage{longtable}
\usepackage{float}
\usepackage{wrapfig}
\usepackage{soul}
\usepackage{textcomp}
\usepackage{marvosym}
\usepackage{wasysym}
\usepackage{latexsym}
\usepackage{amssymb}
\usepackage{hyperref}
\tolerance=1000
\usepackage{verbatim}
\providecommand{\alert}[1]{\textbf{#1}}

\title{Homework 2 Report}
\author{Stephanie Chan}
\date{February 4, 2013}
\hypersetup{
  pdfkeywords={},
  pdfsubject={},
  pdfcreator={Emacs Org-mode version 7.9.3d}}

\begin{document}

\maketitle

\setcounter{tocdepth}{3}
\tableofcontents
\vspace*{1cm}

\section{Create a package for BML grid.}
\label{sec-1}

  The BMLgrid package is in schanBMLgrid. More detail can be found in
  the package manual.
\subsection{Class:}
\label{sec-1-1}

   The BMLgrid class has slots for dimension (the dimension of the
   grid), redcars (the location of the red cars), and bluecars (the
   location of the blue cars).  The BMLgrid is implemented using an S4
   class.  The dimension is a size 2 integer array for the dimension
   of the grid.  The red cars and blue cars is a vector of integers
   determining the location.  The location is given by the single
   integer value of the location of the grid as a matrix.  ie. A
   location at column $x$ and row $y$ would be given by
   \$(x-1)\$*numrows + $y$.
\subsection{Functions:}
\label{sec-1-2}
\subsubsection{generateBMLgrid}
\label{sec-1-2-1}

    This function generates a new BMLgrid with the specified dimension
    and specified number of red cars and blue cars each
\subsubsection{plot}
\label{sec-1-2-2}

    This function is from the general method plot and it creates an
    image plot showing all the red cars and blue cars.
\subsubsection{checkIfCarStuck}
\label{sec-1-2-3}

    This is a helper function used by updateRed and updateBlue to
    check if there is a car in the way before a filled spot
\subsubsection{getdim/redcars/bluecars}
\label{sec-1-2-4}

    These functions return the slots in the BMLgrid class
\subsubsection{updateRed/updateBlue}
\label{sec-1-2-5}

    These functions move the red cars or the blue cars.  In order to
    make these functions faster the numbers are updated by adding or
    subtracting the number of spaces to move right by one column
    (adding the row number) or up by one row (subtracting one).  The
    positions that are at the edges, are taken the modular of in order
    to cycle to the other side of the grid.
\subsubsection{summary}
\label{sec-1-2-6}

    Returns the dimension of the grid and the number of red and blue
    cars on the grid.
\subsubsection{as.data.frame}
\label{sec-1-2-7}

    Returns the cars in the grid object as a data frame with each row
    representing one car and columns of color, row, column of current location.
\subsubsection{checkMoves}
\label{sec-1-2-8}

    Helper function used to check if the cars were moved.
\subsubsection{getVelocity}
\label{sec-1-2-9}

    Function to compute average velocity of all cars by number of cars
    moved divided by the total number of cars.
\subsubsection{updateManySteps}
\label{sec-1-2-10}

    This function will move the car multiple steps.  There is a video
    flag which will display a video of the cars moving.
\section{R profile and timing}
\label{sec-2}

  The time was saved in car movement by moving each of the cars by
  representing each car by a single number to represent the location
  in the matrix and using vectorized numeric operations such as adding
  to move the car.  The \%in\% function was used to check if the car was
  allowed to move.

  The R profile results come from a 100x100 grid with 200 red cars and
  200 blue cars.  These cars were moved through 100 time steps.
\subsection{R profile results}
\label{sec-2-1}


\begin{verbatim}

> summaryRprof("Rprof.out")
$by.self
                  self.time self.pct total.time total.pct
"loadMethod"           0.08    57.14       0.10     71.43
"checkIfCarStuck"      0.02    14.29       0.04     28.57
"assign"               0.02    14.29       0.02     14.29
"match"                0.02    14.29       0.02     14.29

$by.total
                  total.time total.pct self.time self.pct
"standardGeneric"       0.14    100.00      0.00     0.00
"updateManySteps"       0.14    100.00      0.00     0.00
"loadMethod"            0.10     71.43      0.08    57.14
"updateBlue"            0.08     57.14      0.00     0.00
"updateRed"             0.06     42.86      0.00     0.00
"checkIfCarStuck"       0.04     28.57      0.02    14.29
"assign"                0.02     14.29      0.02    14.29
"match"                 0.02     14.29      0.02    14.29
"bluecars"              0.02     14.29      0.00     0.00
"%in%"                  0.02     14.29      0.00     0.00

$sample.interval
[1] 0.02

$sampling.time
[1] 0.14
\end{verbatim}
\section{Checking Velocity as $\rho$ changes}
\label{sec-3}

  We vary the density of cars in the grid and the velocities will
  change with it.  These tests were done on a 100 by 100 grid with car
  densities of $\rho=0.1,0.2,0.3,\ldots,0.9,1.0$, and equal red and
  blue cars.  For $\rho\leq0.3$, the average velocity of the cars tend
  to one, and the traffic jams occur in the beginning.  For
  $\rho>0.5$, the cars are quickly congested and there is little
  velocity and freedom to move.  At $\rho=0.4$ and $\rho=0.5$, the
  trend is also toward getting stuck after enough time passes.
\includegraphics[width=.9\linewidth]{./velocity.pdf}

\clearpage
\section{Appendix}
\label{sec-4}
\subsection{gridS4.R}
\label{sec-4-1}

\verbatiminput{gridS4.R}
\subsection{testgrid.R}
\label{sec-4-2}

\verbatiminput{testgrid.R}

\end{document}
